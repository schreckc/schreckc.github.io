\documentclass{article}
%\usepackage[a4paper, margin=1in]{geometry}
\usepackage{datetime}
\usepackage{wrapfig}
\usepackage[pdftex]{graphicx} \pdfcompresslevel=9
\usepackage[T1]{fontenc}
\usepackage{lmodern}
\usepackage{url}

\usepackage[a4paper, left=2cm,
            right=2cm,
            top=1cm,
            bottom=1.5cm,
            footskip=.5cm]{geometry}
\newdate{date}{31}{01}{2026}
\date{Contact: \texttt{camille.schreck@inria.fr}}

\title{Introduction au Graphisme: Projet Personel}
\author{\displaydate{date}}

\begin{document}


\maketitle
\vspace{-0.8cm}


\begin{figure}[h]
  \centering
  \includegraphics[width=0.7\linewidth]{pentagons}
  \caption{}
  \label{pentagons}
\end{figure}
\vspace{-0.5cm}

\noindent\textbf{Instructions:} Le rendu se fera par mail (camille.schreck@inria.fr). Le sujet du mail devra contenir [PROJET ENSG].
Le projet devra être un fichier appelé \emph{projet\_nom\_prenom.zip} archivant un dossier contenant vos fichiers, ou un fichier \emph{projet\_nom\_prenom.html} si vous n'avez qu'un fichier.\\
\textbf{\textit{Le code devra être commenté.}}

\section*{Pentagons}


\begin{enumerate}
\item Cr\'eer un pentagone centr\'e en z\'ero dans le plan \texttt{(xy)} \`a l'aide de 5 triangles (Figure \ref{penta} gauche).
\item \'Evider l'int\'erieur du petagone \`a l'aide de la commande \texttt{discard} et des coordon\'ees \texttt{uv} (Figure \ref{penta} droite).
\item Faire tourner le pentagone sur lui-m\^eme dans autour de l'axe \texttt{z}.
\end{enumerate}


\begin{figure}[h]
  \centering
  \includegraphics[height=2.5cm]{penta2}
  \hspace{1cm}
  \includegraphics[height=2.5cm]{empty_penta2}
    \caption{}
  \label{penta}
\end{figure}

\section*{Orbites}

\begin{enumerate}
\item Cr\'eer un cercle de \texttt{n} pentagones tournant autour du centre de la sc\`ene  dans le plan \texttt{(xy)} (Figure \ref{orbite}).
\item Faire \'egalement tourner les pentagones sur eux m\^emes.
\item Autour de chacuns des ces pentagons, créer un cercle de \texttt{m} nouveaux pentagones plus petits, de facon \`a ce qu'ils soient entrelacés avec les premiers (voir Figure \ref{pentagons}) et les faire tourner autour.
\end{enumerate}


\begin{figure}[h]
  \centering
  \includegraphics[width=0.6\linewidth]{orbite}
  \caption{}
  \label{orbite}
\end{figure}

\section*{Pentacman}

\begin{enumerate}
\item Dans un nouveau projet, modifier les coordon\'ees \texttt{uv} et le fragment shader de façon à ce que l'un des triangles du pentagon ne conserve que ses deux bords int\'erieurs (voir Figure \ref{pentacman}).
\end{enumerate}

\begin{figure}[h]
  \centering
  \includegraphics[width=0.4\linewidth]{pentacman}
  \caption{}
  \label{pentacman}
\end{figure}

\end{document}

