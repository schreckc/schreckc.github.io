\documentclass{article}
%\usepackage[a4paper, margin=1in]{geometry}
\usepackage{datetime}
\usepackage{wrapfig}
\usepackage[pdftex]{graphicx} \pdfcompresslevel=9
\usepackage[T1]{fontenc}
\usepackage{lmodern}
\usepackage{url}

\usepackage[a4paper, left=2cm,
            right=2cm,
            top=1cm,
            bottom=1.5cm,
            footskip=.5cm]{geometry}
\newdate{date}{31}{01}{2026}
\date{Contact: \texttt{camille.schreck@inria.fr}}

\title{Introduction to Computer Graphics: Personal Project}
\author{\displaydate{date}}

\begin{document}


\maketitle
\vspace{-0.8cm}


\begin{figure}[h]
  \centering
  \includegraphics[width=0.7\linewidth]{pentagons}
  \caption{}
  \label{pentagons}
\end{figure}
\vspace{-0.5cm}

\section*{Pentagons}


\begin{enumerate}
\item Create a pentagon centered on zero in the plane \texttt{(xy)} using 5 triangles (Figure \ref{penta} left).
\item Empty the inside of the pentagon using the \texttt{discard} command and the \texttt{uv} coordinates (Figure \ref{penta} right).
\item Make the pentagon rotate around itself on the \texttt{z} axis.
\end{enumerate}


\begin{figure}[h]
  \centering
  \includegraphics[height=2.5cm]{penta2}
  \hspace{1cm}
  \includegraphics[height=2.5cm]{empty_penta2}
    \caption{}
  \label{penta}
\end{figure}

\section*{Orbites}

\begin{enumerate}
\item Create a circle of \texttt{n} pentagons rotating around the center of the scene in the plane \texttt{(xy)} (Figure \ref{orbite}).
\item Make the pentagons also rotate around themselves.
\item Around each of these pentagons, make a circle of new smaller pentagons such that they are interleaved with the bigger ones (see Figure \ref{pentagons}) and make them rotate around them.
\end{enumerate}


\begin{figure}[h]
  \centering
  \includegraphics[width=0.6\linewidth]{orbite}
  \caption{}
  \label{orbite}
\end{figure}

\section*{Pentacman}

\begin{enumerate}
\item In a new project, modify the \texttt{uv} and the fragment shader such that one of the triangles of the pentagon only keep its two inside borders (see Figure \ref{pentacman}).
\end{enumerate}

\begin{figure}[h]
  \centering
  \includegraphics[width=0.4\linewidth]{pentacman}
  \caption{}
  \label{pentacman}
\end{figure}

\end{document}

